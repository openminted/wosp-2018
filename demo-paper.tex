\documentclass[10pt, a4paper]{article}
\usepackage{lrec}
\usepackage{multibib}
\newcites{languageresource}{Language Resources}
\usepackage{graphicx}
\usepackage{tabularx}
\usepackage{soul}
% for eps graphics

\usepackage{epstopdf}
\usepackage[latin1]{inputenc}

\usepackage{hyperref}
\usepackage{xstring}

\newcommand{\secref}[1]{\StrSubstitute{\getrefnumber{#1}}{.}{ }}

\title{OpenMinTeD: Facilitating Large Scale Text Mining of Open Access Corpora}

\name{Author1, Author2, Author3}

\address{Affiliation1, Affiliation2, Affiliation3 \\
         Address1, Address2, Address3 \\
         author1@xxx.yy, author2@zzz.edu, author3@hhh.com\\
         \{author1, author5, author9\}@abc.org\\}


\abstract{
The OpenMinTeD platform aims to brnig together open-access content from a wide range of publisher and text mining tools from numerous existing natural language processing frameworks in a seamless environment to enable those with little text mining experiance to access the wealth of information locked away in the vast sea of published free-text material.\\ \newline \Keywords{text mining, open access, corpora, natural language processing}}

\begin{document}

\maketitleabstract

\section{Introduction}

\section{Access to Corpora}

Talk about the content conectors and give an idea of how many articles publishers, journals etc. we can give access to for text mining through the platform


\section{Components and Workflows}

we support components from exisiting frameworks (GATE \cite{Cun13a} and UIMA \cite{OASIS:UIMA:2009}), web services, and other tools provided as docker images as long as they follow the guidelines\footnote{\url{https://guidelines.openminted.eu/}}.

these tools are registered with the platform using a common description (OMTD-SHARE) and can then be combined into workflows

workflows are executed via Galaxy on a large cluster to provide scalability and performance

results are pushed back into the OMTD store


\section{Seamless Integration}

The platform serves two roles. Firstly as a text mining expert it's the portal through which you can register components and build workflows (see above), but also it's where none expert users can build corpora and execute pre-existing workflows to mine scholoarly publications for information which they would ont otherwise be able to access.


%\begin{itemize}
%    \item{Compile the \texttt{.tex} file once}
%    \item{Invoke \texttt{bibtex} on the eponymous \texttt{.aux} file}
%    \item{Invoke \texttt{bibtex} on the \texttt{languageresources.aux} file}
%    \item{Compile the \texttt{.tex} file twice}
%\end{itemize}

%\nocite{*}
\section{Bibliographical References}
\label{main:ref}

\bibliographystyle{lrec}
\bibliography{references}


%\section{Language Resource References}
%\label{lr:ref}
%\bibliographystylelanguageresource{lrec}
%\bibliographylanguageresource{references}

\end{document}
