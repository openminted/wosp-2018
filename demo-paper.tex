\documentclass[10pt, a4paper]{article}
\usepackage{lrec}
\usepackage{multibib}
\newcites{languageresource}{Language Resources}
\usepackage{graphicx}
\usepackage{tabularx}
\usepackage{soul}
% for eps graphics

\usepackage{epstopdf}
\usepackage[latin1]{inputenc}

\usepackage{hyperref}
\usepackage{xstring}

\newcommand{\secref}[1]{\StrSubstitute{\getrefnumber{#1}}{.}{ }}

\title{OpenMinTeD: Facilitating Large Scale Text Mining of Open Access Corpora}

\name{Author1, Author2, Author3}

\address{Affiliation1, Affiliation2, Affiliation3 \\
         Address1, Address2, Address3 \\
         author1@xxx.yy, author2@zzz.edu, author3@hhh.com\\
         \{author1, author5, author9\}@abc.org\\}


\abstract{
Each article must include an abstract of 150 to 200 words in Times New Roman
9 with interlinear spacing of 10 pt. The heading Abstract should be
centred, font Times New Roman 10 bold. This short abstract will also be used
for producing the Booklet of Abstracts (PDF) containing the abstracts of all
papers presented at the Conference. \\ \newline \Keywords{keyword1, keyword2,
keyword3} }

\begin{document}

\maketitleabstract

\section{Extended Abstract}

%\begin{itemize}
%    \item{Compile the \texttt{.tex} file once}
%    \item{Invoke \texttt{bibtex} on the eponymous \texttt{.aux} file}
%    \item{Invoke \texttt{bibtex} on the \texttt{languageresources.aux} file}
%    \item{Compile the \texttt{.tex} file twice}
%\end{itemize}

%\nocite{*}
\section{Bibliographical References}
\label{main:ref}

\bibliographystyle{lrec}
%\bibliography{references}


\section{Language Resource References}
\label{lr:ref}
\bibliographystylelanguageresource{lrec}
%\bibliographylanguageresource{references}

\end{document}
